
This chapter summarizes the major problems encountered during system integration, peripheral driver implementation, and the verification process, as well as the solutions adopted and the team’s experiences and suggestions for future improvement.

\subsection{System Integration and Timing}
\begin{itemize}
\item Problem: Each module makes different clock assumptions (for example, \texttt{src/LCD.sv} assumes \texttt{CLK\_HZ=12,000,000}); if not unified, it leads to timing errors with peripherals or initialization failures.
\item Solution: Parameterize all clock-related constants (\texttt{CLK\_HZ}, \texttt{SAMPLES\_PER\_SEC}, etc.) and override them with real board clock values when instantiating in \texttt{Top.sv}; add synchronizers or use FIFO/handshake mechanisms for cross-clock signals; supplement .sdc timing constraints to notify the Timing Analyzer.
\end{itemize}

\subsection{LCD Initialization and Driver Stability}
\begin{itemize}
\item Problem: The LCD either displays garbled text or nothing; typical issues are mismatched timing parameters or insufficient EN/RS pulse lengths.
\item Solution: Confirm the actual clock rate and update \texttt{CLK\_HZ}, or switch from hardcoded microsecond delays to clock tick counting; when necessary, read the Busy Flag (requires making \texttt{o\_LCD\_RW} switchable to input) or keep a conservative delay parameter for easier adjustment; verify timings using an oscilloscope.
\end{itemize}

\subsection{Display Format and Update Efficiency}
\begin{itemize}
\item Problem: Frequently overwriting entire display lines causes unnecessary delays and poor responsiveness.
\item Solution: Only update display data when values change (already implemented by adding a refresh timer and last-value comparison); further optimization can be achieved through per-character updates to reduce write cycles; if the format changes to mm:ss, convert seconds to minutes/seconds before output.
\end{itemize}

\subsection{Sample Rate and Time Conversion}
\begin{itemize}
\item Problem: If the actual sampling rate differs from the programmed rate (e.g., 32k vs 48k), the displayed seconds will be incorrect; values over 999 seconds are truncated.
\item Solution: Set \texttt{SAMPLES\_PER\_SEC} as a top-level parameter and record it in the README; use the mm:ss format for more user-friendly long-duration displays; for values exceeding display limits, show the maximum allowed value or a special string.
\end{itemize}

\subsection{Testing and Verification}
\begin{itemize}
\item Problem: Lack of automated testbench increases the debugging effort on hardware.
\item Solution: Develop a SystemVerilog testbench to simulate initialization, mode switching, and address changes; build unit test sets for pure functions such as \texttt{num\_to\_decimal\_ascii3} (timing boundary and overflow cases).
\end{itemize}

\subsection{Future Improvements}
\begin{enumerate}
\item 
\item 
\item 
\end{enumerate}
